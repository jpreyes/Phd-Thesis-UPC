\documentclass[12pt]{ociamthesis}  %  UPC logo
%%%%%%%%%%%%%%%%%%%%%%%%%%%%%%%%%%%%%%%%%% Disclaimer %%%%%%%%%%%%%%%%%%%%%%%%%%%%%%%%%%%%%%%%%%%%
%This is not an official UPC template. I gathered the resources for my personal use. And I'm happy to share it with anyone who wants to use something out-of-the-box BS/Master/Ph.D latex template. This is a template that based on Oxford/Cambridge LaTeX Phd thesis template with ACL/CVPR confer template inspired modification. 
% original temp github
%https://github.com/mcmanigle/OxThesis
% this temp gitgub
%https://github.com/sabirdvd/LaTeX-template-for-UPC-thesis-
%%%%%%%%%%%%%%%%%%%%%%%%%%%%%%%%%%%%%%%%%% Disclaimer %%%%%%%%%%%%%%%%%%%%%%%%%%%%%%%%%%%%%%%%%%%%

%load any additional packages
\usepackage{amssymb}
\usepackage{graphicx,pstricks}
\usepackage{graphics}
\usepackage{moreverb}
\usepackage{subfigure}
\usepackage{epsfig}
\usepackage{subfigure}
\usepackage{lipsum}  
\usepackage{ragged2e}
\usepackage{hangcaption}
%\usepackage{txfonts}
\usepackage{palatino}
%\usepackage[table]{xcolor}
\usepackage{color}
\usepackage{graphicx}
\usepackage{amsmath,amssymb} % define this before the line numbering.
%\usepackage{ruler}
\usepackage[noend]{algpseudocode}
\usepackage{amsmath,amssymb} % define this before the line numbering.
%\usepackage{ruler}
\usepackage{color}
\usepackage{comment}
\usepackage{bm,nicefrac} % bold math package
\usepackage{amsmath}
\usepackage{tikz}
\DeclareMathOperator*{\argmax}{argmax}
\usepackage{hyperref}
\usepackage{color}
\usepackage{threeparttable}
\usepackage[linesnumbered, ruled]{algorithm2e}
\SetKwRepeat{Do}{do}{while}
\usepackage{graphicx}
\usepackage{color,soul}
%\usepackage{fancyhdr}
%\setlength{\headheight}{15pt}
%\pagestyle{fancy}
%\renewcommand{\chaptermark}[1]{\markboth{#1}{}}
%\renewcommand{\sectionmark}[1]{\markright{#1}{}}
\definecolor{ao}{rgb}{0.0, 0.5, 0.0}

\usepackage{xcolor}
\usepackage{soul}

\newcommand{\hlc}[2][yellow]{{%
    \colorlet{foo}{#1}%
    \sethlcolor{foo}\hl{#2}}%
}


\usepackage{amsmath}
\DontPrintSemicolon
%\usepackage{algorithm}
%\usepackage[linesnumbered,ruled]{algorithm2e}
\usepackage[noend]{algpseudocode}
\usepackage{amsmath,amssymb} % define this before the line numbering.
%\usepackage{ruler}
\usepackage{color}
\usepackage{xcolor}
\definecolor{darkblue}{rgb}{0, 0, 0.5} 
\usepackage{amsmath}
%\DeclareMathOperator*{\argmax}{argmax}
\definecolor{bluegray}{rgb}{0.4, 0.6, 0.8}
\usepackage{hyperref}

%%%%%%%%%%%%%%%%%% Gray box %%%%%%%%%%%%%%%
\usepackage[breakable, theorems, skins]{tcolorbox}
\DeclareRobustCommand{\mybox}[2][gray!20]{%
\begin{tcolorbox}[   %% Adjust the following parameters at will.
        breakable,
        left=0pt,
        right=0pt,
        top=0pt,
        bottom=0pt,
        colback=#1,
        colframe=#1,
        width=\dimexpr\textwidth\relax, 
        enlarge left by=0mm,
        boxsep=5pt,
        arc=0pt,outer arc=0pt,
        ]
        #2
\end{tcolorbox}
}


%

\hypersetup{
  colorlinks   = true, %Colours links instead of ugly boxes
  urlcolor     = black, %Colour for external hyperlinks
  linkcolor    = red, %Colour of internal links RoyalBlue
   % linkcolor    = violet, %Colour of internal links RoyalBlue
   %linkcolor    =  bluegray, %Colour of internal links
  %citecolor   = red %Colour of citations
  %citecolor = blue %black darkblue
  citecolor = darkblue
}
%{ \hypersetup{hidelinks} \tableofcontents }

\hypersetup{linktoc = all}
\hypersetup{linktoc = page} 

\usepackage{lineno,hyperref}
\usepackage{caption} 
\captionsetup[table]{skip=10pt}
\usepackage{color}
\usepackage{graphicx}
\usepackage{amsmath,amssymb} % define this before the line numbering.
%\usepackage{ruler}
\usepackage{color}
\usepackage{threeparttable}
%\usepackage[linesnumbered, ruled]{algorithm2e}
\modulolinenumbers[4]
\usepackage[noend]{algpseudocode}
\usepackage{amsmath,amssymb} % define this before the line numbering.
%\usepackage{ruler}
\usepackage{color}
\usepackage{comment}
\usepackage{bm,nicefrac} % bold math package
\usepackage{amsmath}
\usepackage{tikz}
\usepackage{CJKutf8}
\usepackage{natbib}
\bibpunct{(}{)}{,}{a}{}{;}
\bibliographystyle{abbrvnat}
\renewcommand{\citep}[1]{[\citealp{#1}]}
%\DeclareMathOperator*{\argmax}{argmax}
\usepackage{hyperref}
\usepackage[english]{babel}
\usepackage[utf8]{inputenc}
\usepackage{fancyhdr}
\usepackage{enumitem} 

\usepackage{comment}
%%%%%%%%%%%%%%%%%%%%%%%%%%%%%
%\usepackage{xcolor}
%\usepackage{soulpos}
%\ulposdef{\hlc}[xoffset=1pt]{\mbox{\color{cyan!30}\rule[-.8ex]{\ulwidth}{3ex}}}
%%%%%%%%%%%%%%%%%%%%
\pagestyle{fancy}
\renewcommand{\chaptermark}[1]{\markboth{#1}{#1}}
\fancyhead[R]{}
\fancyhead[L]{\chaptername\ \thechapter\ --\ \leftmark}

\newcommand{\bert}{\ensuremath{%
  \mathchoice{\includegraphics[height=2ex]{Figure-Ch5/Bert.png}} 
    {\includegraphics[height=2ex]{Figure-Ch5/Bert.png}}
    {\includegraphics[height=1.5ex]{Figure-Ch5/Bert.png}}
    {\includegraphics[height=1ex]{Figure-Ch5/Bert.png}}
}} 


\pagestyle{fancy}
\fancyhf{}
\fancyhead[L]{\rightmark}
\fancyhead[R]{\thepage}
%\renewcommand{\headrulewidth}{0pt}
%%%%%%%%%%%%%%%%%%%%%%%%%% from CVPR %%%%%%%%%%%%%%%%%
\makeatletter
\DeclareRobustCommand\onedot{\futurelet\@let@token\@onedot}
\def\@onedot{\ifx\@let@token.\else.\null\fi\xspace}

\def\eg{\emph{e.g}\onedot} \def\Eg{\emph{E.g}\onedot}
\def\eg{\emph{e.g}\onedot} \def\Eg{\emph{E.g}\onedot}
\def\ie{\emph{i.e}\onedot} \def\Ie{\emph{I.e}\onedot}
\def\cf{\emph{c.f}\onedot} \def\Cf{\emph{C.f}\onedot}
\def\etc{\emph{etc}\onedot} \def\vs{\emph{vs}\onedot}
\def\wrt{w.r.t\onedot} \def\dof{d.o.f\onedot}
\def\etal{\emph{et al}\onedot}
%%%%%%%%%%%%%%%%%%%%%%%%%%CVPR %%%%%%%%%%%%%%%%%


%input macros (i.e. write your own macros file called mymacros.tex 
%and uncomment the next line)
%\include{mymacros}


\title{title} 

%\author{sasas }
%\college{Ci\`{e}ncies de la Computaci\'{o}}  %your college

%\college{ {\larger \textit{By}  \vspace{0.3cm} \\ your name} \\  \vspace{0.1cm} Departament de Ci\`{e}ncies de la Computaci\'{o} }

\college{ {\larger \textit{By}  \vspace{0.3cm} \\ your name} }

%\renewcommand{\submittedtext}{change the default text here if needed}
\degree{Doctor of Philosophy}     %the degree
\degreedate{Barcelona, February 2021}     

   
%end the preamble and start the document
\begin{document}



%this baselineskip gives sufficient line spacing for an examiner to easily
%markup the thesis with comments
\baselineskip=18pt plus1pt

%set the number of sectioning levels that get number and appear in the contents
\setcounter{secnumdepth}{3}
\setcounter{tocdepth}{3}


\maketitle                  % create a title page from the preamble info
\begin{dedication}
%\Centering 
%{\csname @flushglue\endcsname=0pt plus .25\textwidth
%\begin{center}



This thesis is submitted to the Computer Science Department, Universitat Polit\`ecnica de Catalunya in fulfilment of the requirements  for the degree in  ... 

%Doctor of Philosophy. This thesis is entirely my own  work, and except where otherwise stated, describes my own research. 



\begin{center}
    

your name, February 2021 
\end{center}

%\null
%\vfill
\begin{center}
    

\vspace*{\fill}%
%\end{center}}

Copyright \textcopyright \space 2021 \\
your name


%All right reserved.
\end{center}



%I hereby declare that except where specific reference is made to the work of others, the
% of this dissertation are original and have not been submitted in whole or in part
%for consideration for any other degree or qualification in this, or any other university.
%This dissertation is my own work and contains nothing which is the outcome of work
%done in collaboration with others, except as specified in the text and Acknowledgements.

%Ahmed Sabir 



\end{dedication}



        % include a dedication.tex file
\begin{acknowledgements}
I would like to thank my 
\end{acknowledgements}
   % include an acknowledgements.tex file
\begin{abstract}
\lipsum[2-2]
\end{abstract}
          % include the abstract

\begin{romanpages}          % start roman page numbering
\tableofcontents            % generate and include a table of contents
\listoffigures              % generate and include a list of figures
\end{romanpages}            % end roman page numbering

%now include the files of latex for each of the chapters etc
\chapter{Introduction}

how to cite .. 
\citep{anderson2018bottom}
\cite{anderson2018bottom}
\citet{anderson2018bottom}
%
\lipsum[2-4]
\chapter{Literature Review}

\lipsum[2-4]
\chapter{Data}

\lipsum[2-4]
\chapter{Learning to Re-rank ..}

\lipsum[2-4]
\chapter{Learning to Re-rank 2}

\lipsum[2-4]
\chapter{Summary and Future work }

\lipsum[2-4]
\chapter{Sample Title}
\label{ch:7}


Lorem ipsum dolor sit amet, consectetur adipiscing elit, sed do eiusmod tempor incididunt ut labore et dolore magna aliqua. Ut enim ad minim veniam, quis nostrud exercitation ullamco laboris nisi ut aliquip ex ea commodo consequat. Duis aute irure dolor in reprehenderit in voluptate velit esse cillum dolore eu fugiat nulla pariatur. Excepteur sint occaecat cupidatat non proident, sunt in culpa qui officia deserunt mollit anim id est laborum.

%now enable appendix numbering format and include any appendices
\appendix
\chapter{The Design of ....}

Throughout this thesis a number of architecture such 



\chapter{}
\lipsum[2-4]

%next line adds the Bibliography to the contents page
\addcontentsline{toc}{chapter}{Bibliography}
%uncomment next line to change bibliography name to references
%\renewcommand{\bibname}{References}
\bibliography{refs}        %use a bibtex bibliography file refs.bib
%\bibliographystyle{plain}  %use the plain bibliography style

\end{document}

